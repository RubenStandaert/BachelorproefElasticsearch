%%=============================================================================
%% Samenvatting
%%=============================================================================

%% TODO: De "abstract" of samenvatting is een kernachtige (~ 1 blz. voor een
%% thesis) synthese van het document.
%%
%% Deze aspecten moeten zeker aan bod komen:
%% - Context: waarom is dit werk belangrijk?
%% - Nood: waarom moest dit onderzocht worden?
%% - Taak: wat heb je precies gedaan?
%% - Object: wat staat in dit document geschreven?
%% - Resultaat: wat was het resultaat?
%% - Conclusie: wat is/zijn de belangrijkste conclusie(s)?
%% - Perspectief: blijven er nog vragen open die in de toekomst nog kunnen
%%    onderzocht worden? Wat is een mogelijk vervolg voor jouw onderzoek?
%%
%% LET OP! Een samenvatting is GEEN voorwoord!

%%---------- Nederlandse samenvatting -----------------------------------------
%%
%% TODO: Als je je bachelorproef in het Engels schrijft, moet je eerst een
%% Nederlandse samenvatting invoegen. Haal daarvoor onderstaande code uit
%% commentaar.
%% Wie zijn bachelorproef in het Nederlands schrijft, kan dit negeren en heel
%% deze sectie verwijderen.

\IfLanguageName{english}{%
\selectlanguage{dutch}
\chapter*{Samenvatting}
\lipsum[1-4]
\selectlanguage{english}
}{}

%%---------- Samenvatting -----------------------------------------------------
%%
%% De samenvatting in de hoofdtaal van het document

\chapter*{\IfLanguageName{dutch}{Samenvatting}{Abstract}}

In dit document staat een onderzoek beschreven naar de belangrijkste voor- en nadelen van het gebruik van
Elasticsearch als tool voor data-analyse. Elasticsearch is een search engine die door veel grote bedrijven actueel gebruikt wordt om data-analyses uit te voeren, om aan logging te doen of om een uitgebreide zoekfunctionaliteit aan te bieden op bijvoorbeeld een webshop. Deze drie functionaliteiten komen terug in het onderzoek maar vooral de data-analyse wordt onder de loep genomen. Dit onderzoek is interessant voor Elasticsearch omdat er in toekomstige updates rekening kan worden gehouden met zwakke punten die in dit onderzoek opduiken. Ook zijn er meer en meer bedrijven die zich baseren op data-analyses om belangrijke beslissingen te nemen. Zulke bedrijven krijgen aan de hand van dit onderzoek een beter inzicht of Elasticsearch een goede keuze is voor hun specifieke use case.

Om de voor-en nadelen van Elasticsearch te onderzoeken wordt er een literatuurstudie gedaan. Een eerste doel van de literatuurstudie is om een verzameling te maken van voor- en nadelen van de search engine die reeds bekend zijn. Een tweede doel van de literatuurstudie is om de technische kant van Elasticsearch onder de loep te nemen. Door dat te doen komen er niet alleen nieuwe voor-en nadelen naar boven maar wordt het uitvoeren van de casus ook ondersteund. De casus houdt in dat er aan de hand van een dataset een paar interessante vragen
opgesteld worden. Vervolgens worden de vragen beantwoord door het opzetten van een project met Elasticsearch.
Door het opzetten en uitvoeren van zo'n project worden nieuwe sterktes en moeilijkheden van Elasticsearch geïntroduceerd. In een ander geval worden er voor-en nadelen die uit de literatuurstudie komen bevestigd of ontkracht. Zo zal er onder andere onderzocht worden of er sprake is van een hoge leercurve en of er voldoende support is. Als laatste wordt er nagegaan welke technieken er worden gebruikt om de analyses zo snel te doen verlopen en welke alternatieven er zijn voor Elasticsearch.

Uit het resultaat zal blijken dat er geen universeel antwoord te vinden is dat geldt voor elke use case. Afhankelijk van de use case zal Elasticsearch andere voor- en nadelen bieden. Daarbij moet men zich afvragen welke eigenschappen het zwaarst doorwegen. Belangrijkste voordelen zijn: real-time search, gemakkelijke installatie, veel support, veel features en sterke schaalbaarheid. Belangrijke nadelen zijn: geen joins tussen verschillende indices en geen vorm van authorizatie.

