%%=============================================================================
%% Conclusie
%%=============================================================================

\chapter{Conclusie}
\label{ch:conclusie}

%% TODO: Trek een duidelijke conclusie, in de vorm van een antwoord op de
%% onderzoeksvra(a)g(en). Wat was jouw bijdrage aan het onderzoeksdomein en
%% hoe biedt dit meerwaarde aan het vakgebied/doelgroep? Reflecteer kritisch
%% over het resultaat. Had je deze uitkomst verwacht? Zijn er zaken die nog
%% niet duidelijk zijn? Heeft het onderzoek geleid tot nieuwe vragen die
%% uitnodigen tot verder onderzoek?

Uit dit onderzoek komen er duidelijk meer voordelen dan nadelen. Het is echter belangrijk om voor een specifieke use case te overwegen welke eigenschappen het meest van belang zijn. Een project opzetten in Elasticsearch blijkt zeer eenvoudig te zijn. De installatie gebeurt over het algemeen in minder dan vijf minuten. Om de data te importeren bieden zowel Elasticsearch als de community verschillende mogelijkheden. Afhankelijk van de mogelijkheid die men kiest verloopt ook dat zeer vlot. Daarna kan de gebruiker meteen aan de slag zonder dat er extra configuratiewerk komt bij kijken. Het gebruik van Elasticsearch kan opgedeeld worden in verschillende luiken. Met het uitgebreid aanbod aan boeken, de gedetailleerde gids, de forums en de steeds groeiende community is er voldoende support om Elasticsearch te leren gebruiken. Om de zoekmachine aan te spreken kan men beroep doen op om het even welke web client of de command line, wat voor veel flexibiliteit zorgt. Er is sprake van een grote leercurve maar tegelijkertijd zijn er enorm veel mogelijkheden. Wanneer men voorkennis heeft van SQL Server wordt deze leercurve al voor een deel verlaagd.

Elasticsearch streeft naar real-time search maar vereenvoudigt andere zaken om dat te bereiken. Men moet zich dus afvragen hoe belangrijk real-time search is voor de use case. Ook moet men er zich van bewust zijn hoe frequent de data verandert. In een omgeving waar de data af en toe verandert is real-time search een haalbaar doel. In een omgeving waar de data zeer frequent verandert kan men soms niet meer van real-time search spreken.

Elasticsearch is ook zeer sterk in schaalbaarheid. Wanneer men werkt met data die veel kan uitgebreid worden heeft men al een groot voordeel als men kiest voor Elasticsearch.

Join-operaties tussen verschillende indices zijn niet mogelijk. Wanneer men een relationele databank zomaar overzet zullen er ongetwijfeld tekortkomingen zijn. Men moet vooraf nadenken welke data men nodig heeft en ervoor zorgen dat die data beschikbaar is binnen eenzelfde index. Dat kan ervoor zorgen dat het importeren van de data toch niet zo vlot verloopt. Wanneer men de data-analyses slechts éénmaal wil uitvoeren is het aanmaken van de verschillende, aangepaste indices veel werk vergeleken met de waarde die men ermee creëert.

Werkt men met gevoelige data, dan kan het zijn dat Elasticsearch toch niet zo interessant is voor de use case. Standaard is er namelijk geen sprake van authorizatie. Iedereen die het systeem gebruikt heeft alle rechten. Wanneer men toch authorizatie nodig heeft zal men moeten betalen voor een extra plugin.

Men kan Elasticsearch volledig gratis gebruiken maar kan het laten hosten voor een prijs vanaf 36,53 EUR.

\subsection{Verder onderzoek}

In dit onderoek werd er onder andere kort gekeken naar enkele alternatieven voor Elasticsearch. Er werden telkens een aantal punten opgesomd waarin Elasticsearch beter scoort dan het alternatief of omgekeerd. Men zou echter veel dieper kunnen gaan in die vergelijking. In verder onderzoek zou zowel de literatuurstudie als het project kunnen herhaald worden voor de alternatieven van Elasticsearch. Aan de hand van die resultaten kan men Elasticsearch op diepgaand niveau vergelijken met de alternatieven. Op die manier kunnen er eventueel nieuwe voor- en nadelen van Elasticsearch naar boven komen die in dit onderzoek niet voorkwamen. Maar vooral wordt het makkelijker voor gebruikers om een gegronde keuze te maken van welke software ze zullen gebruiken voor hun specifieke use case.