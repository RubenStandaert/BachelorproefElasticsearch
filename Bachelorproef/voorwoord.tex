%%=============================================================================
%% Voorwoord
%%=============================================================================

\chapter{Voorwoord}
\label{ch:voorwoord}

%% TODO:
%% Het voorwoord is het enige deel van de bachelorproef waar je vanuit je
%% eigen standpunt (``ik-vorm'') mag schrijven. Je kan hier bv. motiveren
%% waarom jij het onderwerp wil bespreken.
%% Vergeet ook niet te bedanken wie je geholpen/gesteund/... heeft

Het onderzoek in deze bachelorproef werd uitgevoerd in functie van het behalen van mijn diploma Toegepaste Informatica aan de hogeschool HoGent. Het idee voor deze bachelorproef komt van Sabine De Vreese die lector is aan de hogeschool HoGent. Ik kwam bij haar met de vraag of ze enkele interessante onderwerpen kende waarbij data centraal staat. Één van de onderwerpen de ze mij aanreikte was Elasticsearch. Nooit eerder had van Elasticsearch gehoord. Na wat opzoekingswerk werd snel duidelijk dat Elasticsearch door veel grote bedrijven en voor een verscheidenheid aan use cases gebruikt wordt. Het is een zoekmachine die samen met haar community nog steeds aan het groeien is maar ook een aantal alternatieven heeft. Om een breder inzicht te kunnen geven in wanneer men Elasticsearch best verkiest deed ik een onderzoek naar de voor-en nadelen van de zoekmachine. In dat onderzoek mikte ik vooral op data-analyse met Elasticsearch. Bij deze wil ik Sabine De Vreese graag bedanken voor het aanreiken van dit interessante onderwerp. Gedurende het onderzoek heeft Sabine De Vreese ook de rol van promotor op zich genomen. Ze heeft me wekelijks begeleid en bijgestuurd waar nodig. Ook daarvoor wil ik haar graag bedanken. Als laatste wil ik mijn co-promotor Ken Vanderbeken bedanken. Hij bezorgde mij een dataset die ik nodig had om een project in Elasticsearch op te zetten. Naast de dataset kreeg ik ook een aantal vragen die ik in dat project kon oplossen.

