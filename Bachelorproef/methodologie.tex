%%=============================================================================
%% Methodologie
%%=============================================================================

\chapter{Methodologie}
\label{ch:methodologie}

%% TODO: Hoe ben je te werk gegaan? Verdeel je onderzoek in grote fasen, en
%% licht in elke fase toe welke stappen je gevolgd hebt. Verantwoord waarom je
%% op deze manier te werk gegaan bent. Je moet kunnen aantonen dat je de best
%% mogelijke manier toegepast hebt om een antwoord te vinden op de
%% onderzoeksvraag.

Om enkele voor- en nadelen uit de literatuurstudie te bevestigen of te ontkrachten heb ik een project opgezet in Elasticsearch. In het project voerde ik een aantal data-analyses uit met als doel een antwoord te krijgen op enkele interessante vragen. Tijdens het opzetten van het project en het uitvoeren van de data-analyses kon ik bepalen welke voor- en nadelen, die uit de literatuurstudie komen, echt van toepassing zijn. Daarnaast kwam ik ook nieuwe moeilijkheden en sterktes van Elasticsearch tegen die kunnen aansluiten bij het resultaat van de literatuurstudie.

Een voordeel van een project opzetten om bepaalde zaken zelf te gaan ondervinden is dat men de grootte van het voor- of nadeel beter kan inschatten. Als men bijvoorbeeld een zwakte van Elasticsearch ondervind, dan zal men toch een antwoord op de vooraf opgestelde vraag moeten krijgen. Men zal dan ondervinden in hoeverre Elasticsearch toch in staat is om als hulpmiddel te dienen bij het oplossen van de vraag. Of misschien is er workaround waardoor de zwakte eigenlijk niet zo'n groot probleem is. Dat is informatie die met een literatuurstudie moeilijker te verkrijgen is.

\section{De dataset}
De dataset heb ik verkregen van het bedrijf waar ik mijn stage doe. Insites Consulting is een consulting bureau die haar klanten ondersteund in het nemen van doorslaggevende beslissingen. Dat doet men aan de hand van marktonderzoek. Uit de resultaten van zo'n marktonderzoek creërt men dan inzichten die doorgegeven worden aan de klant. 

De ervaring leert dat het soms moeilijk is voor de klant om zo'n inzicht te benutten. In andere gevallen komen de resultaten van het marktonderzoek ergens in een documentje op de computer van de klant en wordt er verder niks mee gedaan. Insites Consulting bouwde enkele jaren geleden een applicatie die dit probleem tegengaat. De applicatie heet Insight Activation Studio. Elke klant krijgt een unieke Studio die aangepast wordt naar de noden, waarden en kleuren van de klant. Die aanpassingen gebeuren in eerste instantie door mensen van Insites Consulting maar ook de klant heeft een dashboard ter beschikking. Een dashboard is een geïsoleerd stuk van de applicatie die admin-gebruikers toelaat om de studio van hun bedrijf te beheren.  

Werknemers van de klant kunnen zich registreren bij de Studio van hun bedrijf. Eens ze aangemeld zijn krijgen ze de mogelijkheid om walls\footnote{Een wall is een verzameling van tiles die min of meer over hetzelfde onderwerp gaan} en tiles\footnote{Een tile is vergelijkbaar met een post op Facebook maar met meer mogelijkheden. Een tile bestaat uit een type (idee, filmpje, foto, observatie, quiz, discussie, …), een titel, een beschrijving en er kan ook media aan toegevoegd worden. Andere mensen kunnen op jouw tile reageren en kunnen deze ook liken.} aan te maken. Ook krijgen ze walls en tiles van anderen te zien. Het is de bedoeling dat de verworven inzichten op deze Studio komen zodat mensen daarop kunnen reageren en liken. Zo ziet men snel welke inzichten populairder zijn bij de werknemers en worden die tot leven gebracht. Uit die inzichten kunnen ook nieuwe ideën vloeien.

Ook de Insights Academy is een feature die in de Studio verweven zit. Daarop kunnen werknemers lessen volgen om te leren hoe men gebruik kan maken van zo’n inzicht. Daarnaast zijn er nog andere zaken (leaderboards, widgets, teams, … ) die het volledige proces leuker maken. Insights Activation Studio is dus eigenlijk de social media voor werknemers binnen een bedrijf om inzichten te creëren en tot leven te brengen.

De dataset die ik gebruikte in mijn project is de SQL Server dataset van de Insight Activation Studio. Een script om de dataset te creëren in SQL Server kunt u vinden in Bijlage 1. \textbf{TODO: BIJLAGE 1 toevoegen} Alle namen van personen die in de dataset voorkomen werden willekeurig gegenereerd om anonimiteit te verzekeren.

\section{De vragen}
De vragen heb ik verkregen van mijn co-promoter, Ken Vanderbeken, die Lead Developer is aan de Insight Activation Studio. Voor hem is het zeer interessant om te weten of er aspecten zijn die bepalen of een tile succesvol is. Een tile stelt een inzicht voor en een inzicht tot leven brengen is net de primaire doelstelling van de applicatie. Daarom gaan alle vragen daarover. Om te bepalen of een tile succesvol is keek ik hoeveel likes en hoeveel comments de tile heeft.

De vragen zijn de volgende:
\begin{itemize}
	\item Heeft de lengte van de titel van een tile invloed op aantal comments en/of likes? 
	\item Heeft de lengte van de description van een tile invloed op aantal comments en/of likes? 
	\item Waar gebeuren de meeste comments en likes? Vanaf de homepage of op wall pages?
	\item Werken bepaalde tile types beter dan andere? Meer response? (like/comment)  
\end{itemize}

\section{Opzetten van het project}


\section{Uitvoeren van de data-analyses}
